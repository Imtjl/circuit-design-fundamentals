В данной лабораторной работе были спроектированы и протестированы два цифровых элемента на КМОП-транзисторах: NOR-вентиль и Демультиплексор <<1 в 4>>. Основной целью было изучить их временные характеристики, измерить задержки распространения, рассчитать максимально допустимые рабочие частоты и проверить реализацию на языке System Verilog.

Для \textbf{NOR-вентиля} задержка распространения \( t_{pd} \) была рассчитана как среднее значение между задержками переднего и заднего фронтов, что составило 12.5 \text{ нс}.
Максимальная частота работы NOR-вентиля составила \( 80 \text{ МГц}. \)
Эти данные подтверждают, что NOR-вентиль на КМОП-транзисторах способен работать на высоких частотах, хотя и имеет ограничения, связанные с физическими характеристиками транзисторов.

Для \textbf{Демультиплексора <<1 в 4>>} задержка распространения была определена как теоретически - как сумма задержек по критическому пути, состоящему из четырёх NOR-вентилей, так и эмпирически - по временной диаграмме. Теоретическое значение \( t_{pd} \) для демультиплексора было рассчитано как \( t_{pd} = 4 \cdot t_{pd_{NOR}} = 50 \text{ нс} \). Однако измерения на временной диаграмме показали более высокую задержку в \(103 \, \text{нс}\), максимальная рабочая частота демультиплексора составила: \(9.7 \text{ МГц}.\)

Эта частота значительно ниже, чем у одиночного NOR-вентиля, что объясняется накоплением задержек на критическом пути из-за последовательного соединения вентилей. Полученные результаты демонстрируют, что производительность комбинационной схемы зависит от критического пути — чем больше элементов в пути, тем выше задержка и ниже рабочая частота.

Для описания и тестирования схемы демультиплексора был применён язык System Verilog, который позволил реализовать модули и тестовое окружение. Тестирование показало корректную работу схемы при всех возможных комбинациях входных сигналов, что подтверждает правильность логики и задержек, заложенных в проект.

Полученные в ходе работы результаты подтверждают необходимость учета задержек при проектировании цифровых схем. Максимальная частота, определяемая как обратная величина к задержке критического пути, является ключевым показателем производительности комбинационных схем. Эти результаты также показывают значимость выбора правильной архитектуры для минимизации задержек, особенно в комплексных схемах. Знания, полученные при работе с Verilog и SPICE-моделированием, будут полезны для дальнейших проектов, требующих учета временных характеристик и обеспечения надёжной работы схем на высоких частотах.
