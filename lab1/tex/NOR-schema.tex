Разработанный NOR-вентиль реализован на КМОП-транзисторах и состоит из пары p-канальных (\( M_1 \), \( M_2 \)) и n-канальных (\( M_3 \), \( M_4 \)) транзисторов. Вентиль выдаёт логическую $"1"$ на выходе \( \text{Out} \), только когда оба входа \( A \) и \( B \) находятся в $"0"$. При подаче $"1"$ на любой из входов nMOS транзисторы подключают выход к земле (GND), устанавливая $"0"$ на выходе, выполняя функцию NOR.

\begin{figure}[H]
	\centering
	\begin{circuitikz}[european, scale=1.5, transform shape]
		\draw (0,0)
		-- (4,0) to[short, *-] (4,0)
		-- (10,0) node[right] {$\small V_{\text{\scriptsize DD}}$};

		\draw (0,-5.5)
		-- (4,-5.5) to[short, *-] (4,-5.5)
		-- (10,-5.5) node[right] {\small GND};

		\draw (4,0) -- (4, -0.23)
		++(0, -0.77) node[pmos,emptycircle]{\small $M_1$} (4, -1.5)
		++ (0, -1.05) node[pmos,emptycircle]{\small $M_2$} (4,-3)
		++(0, -0.325) --(4, -3.41)
		++(0, -1.055) node[nmos,emptycircle]{\small $M_3$} (4, -5)
		++(0, -0.24) -- (4, -5.5);

		\draw
		(1, -1) node[left]{\small $A$}
		-- (3.015, -1);

		\draw (2.5, -1)
		to [short, *-] (2.5, -1)
		-- (2.5, -2.41)
		++ (0, -0.28)
		-- (2.5, -3.55)
		-- (3.861, -3.55) ++(0.139, 0) node[jump crossing] ++(0.14, 0)
		-- (5.3, -3.55)
		-- (5.3, -4.275)
		-- (5.525, -4.275);

		\draw
		(1, -2.55) node[left]{\small $B$}
		-- (2.36, -2.55) ++(0.14, 0) node[jump crossing] (2.55, -2.55)
		++(0.14, 0) -- (3.016, -2.55);

		\draw (2, -2.55)
		to [short, *-] (2, -2.55)
		-- (2, -4.465) -- (3.02, -4.465);

		\draw (6.5, -5.5)
		to [short, *-] (6.5, -5.5)
		-- (6.5, -5.05)
		++(0, 0.775) node[nmos,emptycircle]{\small $M_4$} (6.5, -3.5)
		-- (6.5, -2.75)
		to[short, *-] (6.5, -2.75);

		\draw (5, -2.75)
		-- (10, -2.75) node[right] {\small Out};

		\draw (4, -3.2)
		to[short, *-] (4, -3.2)
		-- (5, -3.2)
		-- (5, -2.75);

	\end{circuitikz}
	\caption{NOR вентиль на nMOS и pMOS транзисторах}
\end{figure}
