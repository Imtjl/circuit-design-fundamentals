\subsubsection{Описание}

Код выполнен на языке \textbf{System Verilog} — расширенном Verilog HDL, созданном в 2005 г. Обозначения $[N:N]$ — это так называемые \textit{векторы}, которые используются для задания многобитных сигналов. В System Verilog векторы представляют собой массивы, каждый бит которых может быть обращен и использован в вычислениях. Например, $[3:0]$ обозначает 4-битный вектор, где биты нумеруются от 3 до 0.

В данном коде представлен демультиплексор 1 к 4, реализованный на уровне логических вентилей. Основная функция демультиплексора заключается в том, чтобы передать входной сигнал \textbf{y} на один из четырёх выходов \textbf{z[3:0]} в зависимости от значения управляющего сигнала \textbf{s[1:0]}.

\subsubsection{Программная реализация на System Verilog}

\begin{lstlisting}
module demultiplexer_1_to_4(
    input logic y,
    input logic [1:0] s,
    output logic [3:0] z
    );

    // first lvl
    wire not_s1, not_s0, not_y;

    nor (not_s0, s[0], s[0]);
    nor (not_s1, s[1], s[1]);
    nor (not_y, y, y);
    
    // second lvl
    wire x1, x2, x3, x4;
    nor(x1, s[0], s[1]);
    nor(x2, not_s0, s[1]);
    nor(x3, s[0], not_s1);
    nor(x4, not_s0, not_s1);
    
    // third lvl
    wire not_x1, not_x2, not_x3, not_x4;
    nor (not_x1, x1, x1);
    nor (not_x2, x2, x2);
    nor (not_x3, x3, x3);
    nor (not_x4, x4, x4);
    
    // fourth lvl
    nor (z[0], not_x1, not_y);
    nor (z[1], not_x2, not_y);
    nor (z[2], not_x3, not_y);
    nor (z[3], not_x4, not_y);
endmodule
\end{lstlisting}
